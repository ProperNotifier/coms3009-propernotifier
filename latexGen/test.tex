\documentclass[12pt]{article}
\usepackage{xcolor}
\usepackage[normalem]{ulem}
\usepackage[utf8]{inputenc}
\usepackage[margin=0.5in]{geometry}
\usepackage[english]{babel}
\usepackage[document]{ragged2e}
\usepackage{ushort}
\begin{document}
	\begin{center}
		\underline{FINITE \quad CALCULUS \quad TUT}\newline
		1. \quad Use \quad  \( k^{3}=k^{\uline{3}}+3k^{\uline{2}}+k^{\uline{1}} \)  \quad to \quad show \quad that\newline
		 \( 0 \leq k < n\sum_{k}3= \frac{n(n-1)}{2} _{(}2 \) \newline
		\noindent\makebox[\linewidth]{\rule{\textwidth}{1pt}}\newline
		2. \quad Use \quad partial \quad fractions \quad to \quad show\newline
		 \( 1 \leq k \leq n\sum \frac{Hk+1-Hk}{k} 1- \frac{Hk+1-Hk1}{kn+1}  \) \newline
		3. \quad Use \quad partial \quad fractions \quad to \quad show\newline
		 \( 1 < k \leq n\sum \)  \quad  \( \frac{H_{k+1} - H_{k}}{k}=2-\frac{1}{n}- \frac{Hn}{n}  \) \newline
		Use \quad falling \quad factorials \quad to \quad evaluate\newline
		4. \quad  \( 1\cdot2+2\cdot3+...+(n-2)\cdot(n-1), \) \newline
		5. \quad  \(  \frac{1}{1\cdot}  \)  \quad + \quad  \quad  \(  \frac{11}{1\cdot2\cdot3} ...+n(n+1) \frac{111}{1\cdot2\cdot3}  \) \newline
		6. \quad  \( k=0\sum^{2} \)  \quad  \( {{k}\choose{m}}={{n+1}\choose{m+1}} \) \newline
		7. \quad Use \quad partial \quad summation \quad to \quad computw \quad  \( 0 \leq k < n\sum \)  \quad  \( H_{k} \) \newline
		8. \quad Compute \quad  \( 1 \leq k \leq n\sum \)  \quad  \( H_{k} \)  \quad by \quad using \quad the \quad double \quad sum \quad  \( 1 \leq i \leq k \leq n\sum \)  \quad  \( \frac{1}{i} \) \newline
		\noindent\makebox[\linewidth]{\rule{\textwidth}{1pt}}\newline
		\underline{Solution \quad to \quad 7.}\newline
	\end{center}
\end{document}